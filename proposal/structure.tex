  Many effective verification tools build on automated solvers.
  %
  These tools reduce problems in an application domain (ranging from
  data-race detection to compiler optimization validation) to the domain of
  a highly optimized solver like Z3.
  %
  However, tool builders rarely formally verify these reductions --- that
  the low-level results from the solver ensure the desired high-level
  property.
  %
  This leaves the soundness of the tool in question.

  A previous paper presented \SpaceSearch, a library to verify such tools within a
  proof assistant.
  %
  A user builds their solver-aided tool in Coq, using operations from the
  \SpaceSearch library to call a solver.
  %
  The user then verifies that the low-level results from solver calls can
  be lifted to the desired high-level properties.
  %
  Once verified, the tool can be extracted (e.g. to Haskell or Racket)
  where the \SpaceSearch interface is instantiated with an efficient
  implementation that uses SMT solver techniques (e.g. Smten or Rosette).

We will evaluate \SpaceSearch by building and verifying an
SMT based tool that checks the correctness of a particular subset of
SQL rewrite rules, and will formally verify it against a semantics of
SQL called \sem~\cite{hottsql}.

\begin{figure}
\begin{small}
\textbf{Rewrite Rule}:
\begin{flalign*}
\quad\; & \SelectFromWhere{\Star}{(\UNIONALL{R}{S})}{\pred} \quad \equiv & \\
        & \UNIONALL{(\SelectFromWhere{\Star}{R}{\pred})} & \\
        & {(\SelectFromWhere{\Star}{S}{\pred})} &
\end{flalign*}

\textbf{\sem Denotation}:
\begin{flalign*}
% \Rightarrow & \lambdaFn{t}{\denote{\UNIONALL{A}{B}} \; t \times \denote{\pred} \; t} \quad= & \\ 
% %
% & \lambdaFn{t}{\denote{\SelectFromWhere{\Star}{A}{\pred}} \; t +
%                \denote{\SelectFromWhere{\Star}{B}{\pred}} \; t} & \\
% %%
\Rightarrow & \lambdaFn{t}{(\denote{R} \; t + \denote{S} \; t) \times \denote{\pred} \; t}
%
\equiv & \\
%
& \lambdaFn{t}{\denote{R} \; t \times \denote{\pred} \; t + \denote{S} \; t \times \denote{\pred} \; t} &
%
\end{flalign*}

\textbf{\sem Proof}:
Apply distributivity of $\times$ over $+$.
\end{small}
\caption{Proving a rewrite rule using \sem.
Recall that \texttt{UNION ALL} means bag-union in SQL, which in \sem
is translated to addition of tuple multiplicities in the two input relations.
}
\label{fig:union-slct}
\end{figure}

The basic datatype in SQL is a {\em relation}, 
which is defined as bag of \emph{tuples}.
Each tuple in the bag must be of the same type, which
is defined by a relation's {\em schema}.
The schema defines the number of elements per tuple, 
and the name and type of each element.

A SQL query takes several input
relations and returns a new output relation. Consider, for example,
a relation $R$, which stores a company's employees and their age. 
$R$ has schema $\{\nameCol : \sqlString; \; \texttt{age} : \sqlInt)\}$
and value $\{(``Alice'', 25),(``Bob'',40),(``Bob'',50)\}$.
%
The SQL query that selects all employees without their age is:
%
\begin{flalign*}
  \SelectFrom{\nameCol}{R}
\end{flalign*}
%
It returns the bag $\{(``Alice''),(``Bob''),(``Bob'')\}$.  

Green \etal's semantics~\cite{k-relations} represent relations as functions that
map every tuple to a natural number that indicates how many times it appears
in the relation. With this semantics,
all queries can be defined in terms of operations on the natural numbers.
We use the notation $\denote{q}$ to define the meaning of
(denote) the SQL query $q$. Consider, for example:

\begin{align*}
  & \denote{\UNIONALL{q_1}{q_2}} =  \lambdaFn{\tuple}{\denote{q_1} \; \tuple+\denote{q_2} \; \tuple} \\ 
  & \denote{\SelectFromWhere{\Star}{q_1}{\pred}} = \lambdaFn{\tuple}{\denote{q_2} \; \tuple \times \denote{\pred} \; \tuple} \\
\end{align*}

The query $\UNIONALL{q_1}{q_2}$ combines 
the tuples in the relation returned by the query $q_1$ with 
the tuples in the relation returned by the query $q_2$ (without removing duplicates).
The result is a relation,
and thus a function, that maps every tuple $\tuple$ to the number of times 
that tuples appears in $q_1$ plus the number of times that the tuple appears in $q_2$.

The query $\SelectFromWhere{\Star}{q}{\pred}$ removes all the tuples, from the
relation returned by the query $q$ for which the predicate $\pred$ does not hold. We denote the predicate
as a function $\denote{\pred}$ that maps a tuple to $1$ if the predicate holds,
and $0$ otherwise. The query multiplies the relation with the
predicate to achieve the desired effect (i.e. $n \times 1 = n$ and $n \times 0 = 0$).

Relations, like the relation $R$ from the first example, can be used as
a based case when nesting the queries defined above.

To prove that two
SQL queries are equal one has to prove that, forall tuples, the two natural number expressions
are equal. 
% which is much simpler than the inductive proof required under the list semantics.
For example, Fig.~\ref{fig:union-slct}
shows how we can prove that selections distribute over unions, by
reducing it to the distributivity of $\times$ over $+$. 
% while
% Fig.~\ref{fig:magic-distinct} shows the proof of the equivalence for
% $\texttt{Q2} \equiv \texttt{Q3}$.

To handle challenges with the denotation of the SQL operation that performs projection,
\sem replaces the natural number in the relation denotation with a homotopic type~\cite{hott-book}, where
the cardinality of the type represents the number of times the tuples appears in the relation. 
%
In this model, the number 
$0$ is the empty type $\zero$,
$1$ is the unit type $\one$,
multiplication is the product type $\times$, and
addition is the sum type $+$.


\begin{figure*}
\centering
\begin{tabular}{|l|l|l|l|l|} \hline
 & Containment (Set)  & Containment (Bag) & Equivalence (Set) &  Equivalence (Bag) \\ \hline 
Conjunctive Queries & NP-Complete \cite{sql-hardness-cq-set} & Open & NP-Complete \cite{sql-hardness-cq-set}  &  Graph Isomorphism \cite{sql-hardness-cq-bag-equality} \\ \hline
% 
Union of Conjunctive Queries & NP-Complete \cite{sql-hardness-cqunion-set}  & Undecidable \cite{sql-hardness-cqunion-bag-containment}  & NP-Complete \cite{sql-hardness-cqunion-set}  & Open \\ \hline 
%
Conjunctive Query with $\neq$, $\geq$, and $\leq$ & 
$\Pi_2^p$-Complete \cite{sql-hardness-cqex-set} & Undecidable  \cite{sql-hardness-cqex-bag} &  $\Pi_2^p$-Complete  \cite{sql-hardness-cqex-set}  & Undecidable \cite{sql-hardness-cqex-bag}  \\ \hline
%
First Order (SQL) Queries & Undecidable \cite{sql-hardness-first-order} & Undecidable  \cite{sql-hardness-first-order} & Undecidable  \cite{sql-hardness-first-order} & Undecidable  \cite{sql-hardness-first-order} \\ \hline
\end{tabular}
\caption{Complexities of Query Containment and Equivalence.}
\label{fig:complexity}
\end{figure*} 

The equivalence of two SQL queries is in general
undecidable~\cite{sql-hardness-first-order}. Figure~\ref{fig:complexity}
shows the complexities of deciding containment and equivalence of
subclasses of SQL.
%
The most well-known subclass are conjunctive queries, which are of the
form \texttt{SELECT DISTINCT $p$ FROM $q$ WHERE $b$}, where $p$ is a
sequence of arbitrarily many attribute projections, $q$ is the cross
product of arbitrarily many input relations, and $b$ is a conjunct
consisting of arbitrarily many equality predicates between attribute
projections.
%
The \sem paper provides a tactic to automatically prove
the equivalence of conjunctive queries in \sem, using Coq's support
for automating deductive reasoning (with \emph{Ltac}).

However, this approach has two major drawbacks. 
1) The tactic is slow because it performs brute-force search over an exponentially large space, and
   because it is implemented in the high-level interpreted Ltac language.
2) The tactic is supposed to be a decision procedure (it only fails if
   the two queries are not equivalent), but there is no proof of this
   property.

We propose to re-implement this tactic as a Coq decision procedure (a
function that either returns a proof that the two queries are equal,
or a proof that they are not equal) using the \SpaceSearch
framework, which will eliminate the drawbacks of the tactic approach:
1) the decision procedure will be fast as it can be extracted to an SMT query, and
2) it will be verified to be a decision procedure.
%
We can potentially also implement a decision procedure for other
subclasses mentioned in~\ref{fig:complexity} that are NP-Complete and
that can thus be reduced to a SAT problem.

By the first checkpoint (November 21), we plan to have implemented the 
algorithm to check the equivalence of conjunctive queries.

By the second checkpoint (December 2), we plan to have verified the 
algorithm to check the equivalence of conjunctive queries.

